% !TeX root = ../../Parte2.tex
\secmeme{css/misplaced}
\section{Primi passi}
\begin{frame}[fragile]{Testo}\transfade\centering
  \begin{description}[<+->]
    \item[colore]
      \begin{tabular}{ll}
        \mintcss{color: red;} & {\color{red} esempio}\\
        \mintcss{color: #0000FF;} & {\color{blue} esempio}
      \end{tabular}
    \bigskip
    \item[dimensione]
      \begin{tabular}{ll}
        \mintcss{font-size: 30pt;} & {\LARGE esempio}
      \end{tabular}
    \bigskip
    \item[carattere]
      \mintcss{font-family: "Times New Roman", Times, serif;}
    \bigskip
    \item[stile]
      \begin{tabular}{ll}
        \mintcss{font-style: italic /* oppure normal */;} & {\textit{esempio}}\\
        \mintcss{font-weight: bold /* oppure normal */;} & {\textbf{esempio}}\\
        \mintcss{text-decoration: underline;} & {\underline{esempio}}\\
        \mintcss{/*oppure overline, line-through, none */}
      \end{tabular}
    \bigskip
    \item[allineamento] \mintcss{text-align: /* left, */ center /*, right, justify */;}
  \end{description}
\end{frame}

\begin{frame}[fragile]{Sfondo}\transfade\centering
  \begin{description}[<+->]
    \item[colore]
      \mintcss{background-color: orange;}
    \bigskip
    \item[immagine]
      \begin{minted}[fontsize=\normalsize]{css}
background-image: url("sfondo.png");
background-repeat: repeat-x /*, repeat-y, no-repeat*/;
background-position: left top;
background-attachment: fixed /*, scroll */;
background-size: cover /*, contain */;
      \end{minted}
    \bigskip
  \end{description}
\end{frame}



%%%%%% MEMORY
\begin{frame}[fragile]\transfade
  \begin{exercise}\centering
    \begin{columns}
      \begin{column}{.45\textwidth}
        \htmlRender[-s A5][height=.85\textheight, trim={0 5cm 0 0},clip]{<meta charset="utf-8"><style>img{width:70px}</style>
    <base href="../../memory/html">
<style>
body{
    background-color: bisque;
    text-align: center;
}

h1{
    font-family: "Comic Sans MS", cursive, sans-serif;
    font-size: 40pt;
    color: blue;
}

.corretto{
    color: green;
}
.errore{
    color: red;
}

#tavolo{
    background-color: white;
}

footer ul a, #gioco button{
    background-color: brown;
    color: white;
    text-align: center;
    text-decoration: none;
}
</style>
<header id="top">
    <hgroup>
        <h1> Memory </h1>
        <h2> Fai la tua mossa </h2>
    </hgroup>
</header>

<article id="gioco">
<table id="tavolo">
      <tr>
          <td><img src="img/dorso.png" alt="Dorso"></td>
          <td><img src="img/dorso.png" alt="Dorso"></td>
          <td><img src="img/dorso.png" alt="Dorso"></td>
          <td><img src="img/dorso.png" alt="Dorso"></td>
      </tr>
      <tr>
          <td><img src="img/dorso.png" alt="Dorso"></td>
          <td><img src="img/dorso.png" alt="Dorso"></td>
          <td><img src="img/dorso.png" alt="Dorso"></td>
          <td><img src="img/dorso.png" alt="Dorso"></td>
      </tr>
      <tr>
          <td><img src="img/dorso.png" alt="Dorso"></td>
          <td><img src="img/dorso.png" alt="Dorso"></td>
          <td><img src="img/dorso.png" alt="Dorso"></td>
          <td><img src="img/dorso.png" alt="Dorso"></td>
      </tr>
      <tr>
          <td><img src="img/dorso.png" alt="Dorso"></td>
          <td><img src="img/dorso.png" alt="Dorso"></td>
          <td><img src="img/dorso.png" alt="Dorso"></td>
          <td><img src="img/dorso.png" alt="Dorso"></td>
      </tr>
  </table>

<section>
<h3 class="corretto"> Giusto </h3>
<h3 class="errore"> Sbagliato </h3>
      <p> Tentativi: 0 </p>
      <p> Giuste: 0 </p>
  </section>

  <button> Ricomincia </button>
</article>

<footer>
    <ul>
        <li> <a href="https://it.wikipedia.org/wiki/Memoria_(gioco)" target="_blank"> Informazioni sul gioco </a>
        <li> <a href="#top"> Torna su </a>
    </ul>
</footer>}
      \end{column}
      \begin{column}{.45\textwidth}
        \begin{enumerate}
          \item Sfondo colorato
          \item Testo allineato al centro
          \item Titolo simpatico e a \texttt{40pt}
          \item Colori per "giusto" / "sbagliato"
          \item Sfondo tavolo bianco
          \item Link e pulsanti con testo bianco centrato su fondo marrone.
          \item Link senza sottolineatura.
        \end{enumerate}
      \end{column}
    \end{columns}
  \end{exercise}
\end{frame}

\begin{frame}[fragile]\transfade
  \begin{sol}\centering
    \texttt{memory.html}:
    \mint{html}{<link rel="stylesheet" href="memory.css">}
    \begin{enumerate}
      \item Sfondo colorato
      \item Testo allineato al centro\\
      \texttt{memory.css}:
      \makeatletter
      \inputminted[breaklines, firstline=4, lastline=7]{css}{\html@dir/\jobname _\thehtml@count.html}
      \item Titolo simpatico e a \texttt{40pt}
      \inputminted[breaklines, firstline=9, lastline=13]{css}{\html@dir/\jobname _\thehtml@count.html}
      \makeatother
    \end{enumerate}
  \end{sol}
\end{frame}

\begin{frame}[fragile]\transfade
  \begin{sol}\centering
    \begin{enumerate}
      \setcounter{enumi}{3}
      \item Colori per "giusto" / "sbagliato"\\
      \texttt{memory.html}:
      \makeatletter
      \inputminted[breaklines, firstline=69, lastline=70]{html}{\html@dir/\jobname _\thehtml@count.html}
      \texttt{memory.css}:
      \inputminted[breaklines, firstline=15, lastline=20]{css}{\html@dir/\jobname _\thehtml@count.html}
      \item Sfondo tavolo bianco\\
      \texttt{memory.html}:
      \makeatletter
      \inputminted[breaklines, firstline=41, lastline=41]{html}{\html@dir/\jobname _\thehtml@count.html}
      \texttt{memory.css}:
      \inputminted[breaklines, firstline=22, lastline=24]{css}{\html@dir/\jobname _\thehtml@count.html}
      \makeatother
    \end{enumerate}
  \end{sol}
\end{frame}

\begin{frame}[fragile]\transfade
  \begin{sol}\centering
    \begin{enumerate}
      \setcounter{enumi}{5}
      \item Link e pulsanti con testo bianco centrato su fondo marrone.\\
      \texttt{memory.html}:
      \makeatletter
      \inputminted[breaklines, firstline=40, lastline=40]{html}{\html@dir/\jobname _\thehtml@count.html}
      \texttt{memory.css}:
      \begin{minted}{css}
footer ul a, #gioco button{
    background-color: brown;
    color: white;
    text-align: center;
}
\end{minted}
      \item Link senza sottolineatura.
      \inputminted[breaklines, firstline=26, lastline=31]{css}{\html@dir/\jobname _\thehtml@count.html}
      \makeatother
    \end{enumerate}
  \end{sol}
\end{frame}