% !TeX root = ../../Parte2.tex
\secmeme{css/misure}
\section{Media queries}

\begin{frame}[fragile]{Media queries}\transfade\centering
  \begin{minted}{css}
    @media screen and (max-width: 400px) {
        body {
            background-color: #FFFF00;
        }
    }
  \end{minted}  
\end{frame}

\begin{frame}[fragile]{Tipi di dispositivo}\transfade\centering
  \begin{description}[<+->]
    \itemtt[\mintcss{@media all}] Tutti i dispositivi.
    \itemtt[\mintcss{@media screen}] Sullo schermo (computer, telefono, tablet, televisione, \dots).
    \itemtt[\mintcss{@media print}] Per la stampa.
    \itemtt[\mintcss{@media speech}] Per i lettori di schermo.
    \bigskip    
    \itemtt[\mintcss{@media not screen}] Per tutti tranne gli schermi.
  \end{description}
\end{frame}

\begin{frame}[fragile]{Proprietà del dispositivo}\transfade\centering
  \begin{minted}{css}
    @media screen and (min-width: 200px and max-width: 400px) {
        /* per finestre larghe fra i 200px e i 400px */
    }
    
    @media screen and (min-height: 200px and max-height: 400px) {
        /* per finestre alte fra i 200px e i 400px */
    }
    
    @media screen and (orientation: landscape /* oppure portrait */) {
        /* se la visuale è orizzontale (oppure verticale) */
        /* anche sui computer, non solo sui telefoni e tablet */
    }
  \end{minted}  
\end{frame}
