% !TeX root = ../../Parte2.tex
\secmeme{css/float}
\section[Posizioni]{Posizionamento degli elementi}

\begin{frame}{Tipi di elemento}\transfade\centering
  \begin{description}
    \item[block] Inizia in una nuova linea e occupa tutta la larghezza disponibile, es:
      \begin{itemize}
        \item \minttag{h1}, ..., \minttag{h6}
        \item \minttag{p}
        \item \minttag{heder}, \minttag{footer}
        \item \minttag{section}, \minttag{article}
        \item \minttag{table}
        \item \minttag{ul}, \minttag{ol}, \minttag{li}
        \item \minttag{div}
        \item \minttag{form}
        \item \minttag{figure}
      \end{itemize}
    \item[inline] Non va a capo e occupa solo la larghezza necessaria, es:
    \begin{itemize}
      \item \minttag{span}
      \item \minttag{a}
      \item \minttag{img}
      \item \minttag{input}
    \end{itemize}
  \end{description}
\end{frame}

\begin{frame}{Cambiare la visualizzazione di un elemento}\transfade\centering
  \begin{itemize}
    \item\mintcss{display: block;}
    \item\mintcss{display: inline;}
    \item\mintcss{display: none;} Fa sparire l'elemento, come se non fosse presente.
    \item\mintcss{visibility: hidden;} Rende invisibile l'elemento, ma mantiene occupato lo spazio.
  \end{itemize}
\end{frame}

\begin{frame}[fragile]{Centrare orizzontalmente}\transfade\centering
  \htmlDual[-s A7][][trim={0 9cm 0 0},clip][css]{<style>}{
#contenitore{ /*padre*/
  border: 1px solid blue;
}
#contenuto{ /*figlio*/
  border: 1px solid orange;
  width: 200px;
  margin: auto;
}
  }{
  </style>
  <div id="contenitore"><p id="contenuto">Lorem ipsum dolor sit amen.</p></div>}
  \bigskip
  Funziona solo per i \mintcss{#contenuto} di tipo \emph{block} (non \emph{inline}).\\
\end{frame}

\begin{frame}[fragile]{Float}\transfade\centering
  \htmlDual[-s A7][breaklines][trim={0 3cm 0 0},clip]{<meta charset="utf-8"><base href="../"><style>img{width:200px}</style>}{
<img id="fig" src="disegno.png">Lorem ipsum dolor sit amet, consectetur adipiscing elit. Quisque consequat sagittis dapibus. In faucibus risus ut suscipit venenatis. Vestibulum sed urna tortor.
  }{}
\end{frame}
\begin{frame}[fragile]{Float}\transfade\centering
  \htmlDual[-s A7][fontsize=\noexpand\normalsize][trim={0 5.2cm 0 .5cm},clip][css]{<meta charset="utf-8"><base href="../"><style>img{width:200px}}{
#fig{
  float: left;
}
  }{</style>
<img id="fig" src="disegno.png">Lorem ipsum dolor sit amet, consectetur adipiscing elit. Quisque consequat sagittis dapibus. In faucibus risus ut suscipit venenatis. Vestibulum sed urna tortor.
  }
  \htmlDual[-s A7][fontsize=\noexpand\normalsize][trim={0 5.2cm 0 .5cm},clip][css]{<meta charset="utf-8"><base href="../"><style>img{width:200px}}{
#fig{
  float: right;
}
  }{</style>
<img id="fig" src="disegno.png">Lorem ipsum dolor sit amet, consectetur adipiscing elit. Quisque consequat sagittis dapibus. In faucibus risus ut suscipit venenatis. Vestibulum sed urna tortor.
  }
  \mintcss{float: none} come di default\\
\end{frame}

\begin{frame}{Tipi di posizionamento}\transfade\centering
  \begin{description}
    \item[\mintcss{position: static;}] Default.
    \item[\mintcss{position: realtive;}] Mediante le proprietà \mintcss{top}, \mintcss{bottom}, \mintcss{left} e \mintcss{right} può essere spostato in relazione alla suo posizione di default.
    \item[\mintcss{position: fixed;}] Mediante le proprietà \mintcss{top}, \mintcss{bottom}, \mintcss{left} e \mintcss{right} può essere posizionato in relazione alla pagina. Non si muove allo scorrere della pagina.
    \item[\mintcss{position: sticky;}] Come \mintcss{relative} quando rientra nello schermo, diventa \mintcss{fixed} quando scorrendo la pagina uscirebbe dallo schermo.
    \item[\mintcss{position: absolute;}] Mediante le proprietà \mintcss{top}, \mintcss{bottom}, \mintcss{left} e \mintcss{right} può essere posizionato in relazione al padre (conteniotre).
  \end{description}
\end{frame}

% !TeX root = ../../Parte2.tex
\secmeme{css/flex}
\subsection[Flexbox]{Introduzione alle flexbox}

\begin{frame}[fragile]{Contenitore Flex}\transfade\centering
  \htmlDual[-s A7][][trim={0 8cm 0 0},clip]{<style>*{visibility:hidden}div,section{visibility:visible}</style>}{
<style>
    #contenitore{
        display: flex;
        border: red solid;
    }
    #contenitore div{
        padding: 5px;
        margin:5px;
        border: green solid;
    }
</style>

...

<section id="contenitore">
    <div>A</div>
    <div>B</div>
    <div>C</div>
</section>
    }{}
\end{frame}

\begin{frame}[fragile]{Contenitori riga e colonna}\transfade\centering
  \htmlDual[-s A7][][trim={0 8cm 0 0},clip][css]{<style>}{
#contenitore{ /* riga */
    display: flex;
    border: red solid;
    flex-direction: row;
}
}{
    #contenitore div{
        padding: 5px;
        margin:5px;
        border: green solid;
    }
</style>

<section id="contenitore">
    <div>A</div>
    <div>B</div>
    <div>C</div>
</section>
    }
  \htmlDual[-s A7][][trim={0 7cm 0 0},clip][css]{<style>}{
#contenitore{ /* colonna */
    display: flex;
    border: red solid;
    flex-direction: column;
}
}{
      #contenitore div{
          padding: 5px;
          margin:5px;
          border: green solid;
      }
  </style>

  <section id="contenitore">
      <div>A</div>
      <div>B</div>
      <div>C</div>
  </section>
      }
\end{frame}

\begin{frame}[fragile]{Responsive wrap}\transfade\centering
  \htmlDual[-s A7][][trim={0 8cm 0 0},clip][css]{<style>}{
#contenitore{
    display: flex;
    border: red solid;
}
}{
    #contenitore div{
        padding: 5px;
        margin:5px;
        border: green solid;
    }
</style>

<section id="contenitore">
<div>A</div>
<div>B</div>
<div>C</div>
<div>D</div>
<div>E</div>
<div>F</div>
<div>G</div>
<div>H</div>
<div>I</div>
</section>
    }
  \htmlDual[-s A7][][trim={0 6cm 0 0},clip][css]{<style>}{
#contenitore{
    display: flex;
    border: red solid;
    flex-wrap: wrap;
}
}{
      #contenitore div{
          padding: 5px;
          margin:5px;
          border: green solid;
      }
  </style>

  <section id="contenitore">
  <div>A</div>
<div>B</div>
<div>C</div>
<div>D</div>
<div>E</div>
<div>F</div>
<div>G</div>
<div>H</div>
<div>I</div>
<div>J</div>
<div>K</div>
<div>L</div>
<div>M</div>
<div>N</div>
<div>O</div>
<div>P</div>
<div>Q</div>
<div>R</div>
<div>S</div>
<div>T</div>
<div>U</div>
<div>V</div>
<div>W</div>
<div>X</div>
<div>Y</div>
<div>Z</div>
  </section>
      }
\end{frame}

\begin{frame}[fragile]{Allineamento principale semplice}\transfade\centering
  \htmlDual[-s A7][][trim={0 8cm 0 0},clip][css]{<style>}{
#contenitore{
    display: flex;
    border: red solid;
    justify-content: center;
}
}{#contenitore div{padding: 5px;margin:5px;border: green solid;}</style><section id="contenitore"><div>A</div><div>B</div><div>C</div></section>}
  \htmlDual[-s A7][][trim={0 8cm 0 0},clip][css]{<style>}{
#contenitore{
    display: flex;
    border: red solid;
    justify-content: flex-start;
}
}{#contenitore div{padding: 5px;margin:5px;border: green solid;}</style><section id="contenitore"><div>A</div><div>B</div><div>C</div></section>}
  \htmlDual[-s A7][][trim={0 8cm 0 0},clip][css]{<style>}{
#contenitore{
    display: flex;
    border: red solid;
    justify-content: flex-end;
}
}{#contenitore div{padding: 5px;margin:5px;border: green solid;}</style><section id="contenitore"><div>A</div><div>B</div><div>C</div></section>}
\end{frame}

\begin{frame}[fragile]{Allineamento principale con distanziamento}\transfade\centering
  \htmlDual[-s A7][][trim={0 8cm 0 0},clip][css]{<style>}{
#contenitore{
    display: flex;
    border: red solid;
    justify-content: space-around;
}
}{#contenitore div{padding: 5px;margin:5px;border: green solid;}</style><section id="contenitore"><div>A</div><div>B</div><div>C</div></section>}
  \htmlDual[-s A7][][trim={0 8cm 0 0},clip][css]{<style>}{
#contenitore{
    display: flex;
    border: red solid;
    justify-content: space-between;
}
}{#contenitore div{padding: 5px;margin:5px;border: green solid;}</style><section id="contenitore"><div>A</div><div>B</div><div>C</div></section>}

\htmlDual[-s A7][][trim={0 5cm 0 0},clip][css]{<style>}{
  #contenitore{
      display: flex;
      border: red solid;
      height: 200px;
      flex-direction: column;
      justify-content: space-around;
  }
  }{#contenitore div{padding: 5px;margin:5px;border: green solid;}</style><section id="contenitore"><div>A</div><div>B</div><div>C</div></section>}
\end{frame}

\begin{frame}[fragile]{Allineamento secondario}\transfade\centering
  \htmlDual[-s A7][][trim={0 8cm 0 0},clip][css]{<style>}{
#contenitore{
    display: flex;
    border: red solid;
    height: 70px;
}
}{#contenitore div{padding: 5px;margin:5px;border: green solid;}</style><section id="contenitore"><div>A</div><div>B</div><div>C</div></section>}
  \htmlDual[-s A7][][trim={0 8cm 0 0},clip][css]{<style>}{
#contenitore{
    display: flex;
    height: 70px;
    align-items: center;
}
}{#contenitore div{padding: 5px;margin:5px;border: green solid;}</style><section style="border: red solid" id="contenitore"><div>A</div><div>B</div><div>C</div></section>}
\htmlDual[-s A7][][trim={0 8cm 0 0},clip][css]{<style>}{
#contenitore{
  display: flex;
  height: 70px;
  align-items: flex-start;
}
}{#contenitore div{padding: 5px;margin:5px;border: green solid;}</style><section style="border: red solid" id="contenitore"><div>A</div><div>B</div><div>C</div></section>}
\htmlDual[-s A7][][trim={0 8cm 0 0},clip][css]{<style>}{
#contenitore{
  display: flex;
  height: 70px;
  align-items: flex-end;
}
}{#contenitore div{padding: 5px;margin:5px;border: green solid;}</style><section style="border: red solid" id="contenitore"><div>A</div><div>B</div><div>C</div></section>}
\end{frame}