% !TeX root = ../../Parte2.tex
\secmeme{css/float}
\section[Posizioni]{Posizionamento degli elementi}

\begin{frame}{Tipi di elemento}\transfade\centering
  \begin{description}
    \item[block] Inizia in una nuova linea e occupa tutta la larghezza disponibile, es:
      \begin{itemize}
        \item \minttag{h1}, ..., \minttag{h6}
        \item \minttag{p}
        \item \minttag{heder}, \minttag{footer}
        \item \minttag{section}, \minttag{article}
        \item \minttag{table}
        \item \minttag{ul}, \minttag{ol}, \minttag{li}
        \item \minttag{div}
        \item \minttag{form}
        \item \minttag{figure}
      \end{itemize}
    \item[inline] Non va a capo e occupa solo la larghezza necessaria, es:
    \begin{itemize}
      \item \minttag{span}
      \item \minttag{a}
      \item \minttag{img}
      \item \minttag{input}
    \end{itemize}
  \end{description}
\end{frame}

\begin{frame}{Cambiare la visualizzazione di un elemento}\transfade\centering
  \begin{itemize}
    \item\mintcss{display: block;}
    \item\mintcss{display: inline;}
    \item\mintcss{display: none;} Fa sparire l'elemento, come se non fosse presente.
    \item\mintcss{visibility: hidden;} Rende invisibile l'elemento, ma mantiene occupato lo spazio.
  \end{itemize}
\end{frame}

\begin{frame}[fragile]{Centrare orizzontalmente}\transfade\centering
  \htmlDual[-s A7][][trim={0 9cm 0 0},clip][css]{<style>}{
#contenitore{ /*padre*/
  border: 1px solid blue;
}
#contenuto{ /*figlio*/
  border: 1px solid orange;
  width: 200px;
  margin: auto;
}
  }{
  </style>
  <div id="contenitore"><p id="contenuto">Lorem ipsum dolor sit amen.</p></div>}
  \bigskip
  Funziona solo per i \mintcss{#contenuto} di tipo \emph{block} (non \emph{inline}).\\
\end{frame}

\begin{frame}[fragile]{Float}\transfade\centering
  \htmlDual[-s A7][breaklines][trim={0 3cm 0 0},clip]{<meta charset="utf-8"><base href="../"><style>img{width:200px}</style>}{
<img id="fig" src="disegno.png">Lorem ipsum dolor sit amet, consectetur adipiscing elit. Quisque consequat sagittis dapibus. In faucibus risus ut suscipit venenatis. Vestibulum sed urna tortor.
  }{}
\end{frame}
\begin{frame}[fragile]{Float}\transfade\centering
  \htmlDual[-s A7][fontsize=\noexpand\normalsize][trim={0 5.2cm 0 .5cm},clip][css]{<meta charset="utf-8"><base href="../"><style>img{width:200px}}{
#fig{
  float: left;
}
  }{</style>
<img id="fig" src="disegno.png">Lorem ipsum dolor sit amet, consectetur adipiscing elit. Quisque consequat sagittis dapibus. In faucibus risus ut suscipit venenatis. Vestibulum sed urna tortor.
  }
  \htmlDual[-s A7][fontsize=\noexpand\normalsize][trim={0 5.2cm 0 .5cm},clip][css]{<meta charset="utf-8"><base href="../"><style>img{width:200px}}{
#fig{
  float: right;
}
  }{</style>
<img id="fig" src="disegno.png">Lorem ipsum dolor sit amet, consectetur adipiscing elit. Quisque consequat sagittis dapibus. In faucibus risus ut suscipit venenatis. Vestibulum sed urna tortor.
  }
  \mintcss{float: none} come di default\\
\end{frame}

\begin{frame}{Tipi di posizionamento}\transfade\centering
  \begin{description}
    \item[\mintcss{position: static;}] Default.
    \item[\mintcss{position: realtive;}] Mediante le proprietà \mintcss{top}, \mintcss{bottom}, \mintcss{left} e \mintcss{right} può essere spostato in relazione alla suo posizione di default.
    \item[\mintcss{position: fixed;}] Mediante le proprietà \mintcss{top}, \mintcss{bottom}, \mintcss{left} e \mintcss{right} può essere posizionato in relazione alla pagina. Non si muove allo scorrere della pagina.
    \item[\mintcss{position: sticky;}] Come \mintcss{relative} quando rientra nello schermo, diventa \mintcss{fixed} quando scorrendo la pagina uscirebbe dallo schermo.
    \item[\mintcss{position: absolute;}] Mediante le proprietà \mintcss{top}, \mintcss{bottom}, \mintcss{left} e \mintcss{right} può essere posizionato in relazione al padre (conteniotre).
  \end{description}
\end{frame}

\subsection{Flexbox}

\begin{frame}{Flex}\transfade\centering

\end{frame}