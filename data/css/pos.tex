% !TeX root = ../../Parte2.tex
\secmeme{css/float}
\section[Posizioni]{Posizionamento degli elementi}

\begin{frame}{Tipi di elemento}\transfade\centering
  \begin{description}
    \item[block] Inizia in una nuova linea e occupa tutta la larghezza disponibile, es:
      \begin{itemize}
        \item \minttag{h1}, ..., \minttag{h6}
        \item \minttag{p}
        \item \minttag{heder}, \minttag{footer}
        \item \minttag{section}, \minttag{article}
        \item \minttag{table}
        \item \minttag{ul}, \minttag{ol}, \minttag{li}
        \item \minttag{div}
        \item \minttag{form}
        \item \minttag{figure}
      \end{itemize}
    \item[inline] Non va a capo e occupa solo la larghezza necessaria, es:
    \begin{itemize}
      \item \minttag{span}
      \item \minttag{a}
      \item \minttag{img}
      \item \minttag{input}
    \end{itemize}
  \end{description}
\end{frame}

\begin{frame}{Cambiare la visualizzazione di un elemento}\transfade\centering
  \begin{itemize}
    \item\mintcss{display: block;}
    \item\mintcss{display: inline;}
    \item\mintcss{display: none;} Fa sparire l'elemento, come se non fosse presente.
    \item\mintcss{visibility: hidden;} Rende invisibile l'elemento, ma mantiene occupato lo spazio.
  \end{itemize}
\end{frame}

\begin{frame}[fragile]{Centrare orizzontalmente}\transfade\centering
  \htmlDual[-s A7][][trim={0 9cm 0 0},clip][css]{<style>}{
#contenitore{ /*padre*/
  border: 1px solid blue;
}
#contenuto{ /*figlio*/
  border: 1px solid orange;
  width: 200px;
  margin: auto;
}
  }{
  </style>
  <div id="contenitore"><p id="contenuto">Lorem ipsum dolor sit amen.</p></div>}
  \bigskip
  Funziona solo per i \mintcss{#contenuto} di tipo \emph{block} (non \emph{inline}).\\
\end{frame}

\begin{frame}[fragile]{Float}\transfade\centering
  \htmlDual[-s A7][breaklines][trim={0 3cm 0 0},clip]{<meta charset="utf-8"><base href="../"><style>img{width:200px}</style>}{
<img id="fig" src="disegno.png">Lorem ipsum dolor sit amet, consectetur adipiscing elit. Quisque consequat sagittis dapibus. In faucibus risus ut suscipit venenatis. Vestibulum sed urna tortor.
  }{}
\end{frame}
\begin{frame}[fragile]{Float}\transfade\centering
  \htmlDual[-s A7][fontsize=\noexpand\normalsize][trim={0 5.2cm 0 .5cm},clip][css]{<meta charset="utf-8"><base href="../"><style>img{width:200px}}{
#fig{
  float: left;
}
  }{</style>
<img id="fig" src="disegno.png">Lorem ipsum dolor sit amet, consectetur adipiscing elit. Quisque consequat sagittis dapibus. In faucibus risus ut suscipit venenatis. Vestibulum sed urna tortor.
  }
  \htmlDual[-s A7][fontsize=\noexpand\normalsize][trim={0 5.2cm 0 .5cm},clip][css]{<meta charset="utf-8"><base href="../"><style>img{width:200px}}{
#fig{
  float: right;
}
  }{</style>
<img id="fig" src="disegno.png">Lorem ipsum dolor sit amet, consectetur adipiscing elit. Quisque consequat sagittis dapibus. In faucibus risus ut suscipit venenatis. Vestibulum sed urna tortor.
  }
  \mintcss{float: none} come di default\\
\end{frame}

\begin{frame}{Tipi di posizionamento}\transfade\centering
  \begin{description}
    \item[\mintcss{position: static;}] Default.
    \item[\mintcss{position: realtive;}] Mediante le proprietà \mintcss{top}, \mintcss{bottom}, \mintcss{left} e \mintcss{right} può essere spostato in relazione alla suo posizione di default.
    \item[\mintcss{position: fixed;}] Mediante le proprietà \mintcss{top}, \mintcss{bottom}, \mintcss{left} e \mintcss{right} può essere posizionato in relazione alla pagina. Non si muove allo scorrere della pagina.
    \item[\mintcss{position: sticky;}] Come \mintcss{relative} quando rientra nello schermo, diventa \mintcss{fixed} quando scorrendo la pagina uscirebbe dallo schermo.
    \item[\mintcss{position: absolute;}] Mediante le proprietà \mintcss{top}, \mintcss{bottom}, \mintcss{left} e \mintcss{right} può essere posizionato in relazione al padre (conteniotre).
  \end{description}
\end{frame}

% !TeX root = ../../Parte2.tex
\secmeme{css/flex}
\subsection[Flexbox]{Introduzione alle flexbox}

\begin{frame}[fragile]{Contenitore Flex}\transfade\centering
  \htmlDual[-s A7][][trim={0 8cm 0 0},clip]{<style>*{visibility:hidden}div,section{visibility:visible}</style>}{
<style>
    #contenitore{
        display: flex;
        border: red solid;
    }
    #contenitore div{
        padding: 5px;
        margin:5px;
        border: green solid;
    }
</style>

...

<section id="contenitore">
    <div>A</div>
    <div>B</div>
    <div>C</div>
</section>
    }{}
\end{frame}

\begin{frame}[fragile]{Contenitori riga e colonna}\transfade\centering
  \htmlDual[-s A7][][trim={0 8cm 0 0},clip][css]{<style>}{
#contenitore{ /* riga */
    display: flex;
    border: red solid;
    flex-direction: row;
}
}{
    #contenitore div{
        padding: 5px;
        margin:5px;
        border: green solid;
    }
</style>

<section id="contenitore">
    <div>A</div>
    <div>B</div>
    <div>C</div>
</section>
    }
  \htmlDual[-s A7][][trim={0 7cm 0 0},clip][css]{<style>}{
#contenitore{ /* colonna */
    display: flex;
    border: red solid;
    flex-direction: column;
}
}{
      #contenitore div{
          padding: 5px;
          margin:5px;
          border: green solid;
      }
  </style>

  <section id="contenitore">
      <div>A</div>
      <div>B</div>
      <div>C</div>
  </section>
      }
\end{frame}

\begin{frame}[fragile]{Responsive wrap}\transfade\centering
  \htmlDual[-s A7][][trim={0 8cm 0 0},clip][css]{<style>}{
#contenitore{
    display: flex;
    border: red solid;
}
}{
    #contenitore div{
        padding: 5px;
        margin:5px;
        border: green solid;
    }
</style>

<section id="contenitore">
<div>A</div>
<div>B</div>
<div>C</div>
<div>D</div>
<div>E</div>
<div>F</div>
<div>G</div>
<div>H</div>
<div>I</div>
</section>
    }
  \htmlDual[-s A7][][trim={0 6cm 0 0},clip][css]{<style>}{
#contenitore{
    display: flex;
    border: red solid;
    flex-wrap: wrap;
}
}{
      #contenitore div{
          padding: 5px;
          margin:5px;
          border: green solid;
      }
  </style>

  <section id="contenitore">
  <div>A</div>
<div>B</div>
<div>C</div>
<div>D</div>
<div>E</div>
<div>F</div>
<div>G</div>
<div>H</div>
<div>I</div>
<div>J</div>
<div>K</div>
<div>L</div>
<div>M</div>
<div>N</div>
<div>O</div>
<div>P</div>
<div>Q</div>
<div>R</div>
<div>S</div>
<div>T</div>
<div>U</div>
<div>V</div>
<div>W</div>
<div>X</div>
<div>Y</div>
<div>Z</div>
  </section>
      }
\end{frame}

\begin{frame}[fragile]{Allineamento principale semplice}\transfade\centering
  \htmlDual[-s A7][][trim={0 8cm 0 0},clip][css]{<style>}{
#contenitore{
    display: flex;
    border: red solid;
    justify-content: center;
}
}{#contenitore div{padding: 5px;margin:5px;border: green solid;}</style><section id="contenitore"><div>A</div><div>B</div><div>C</div></section>}
  \htmlDual[-s A7][][trim={0 8cm 0 0},clip][css]{<style>}{
#contenitore{
    display: flex;
    border: red solid;
    justify-content: flex-start;
}
}{#contenitore div{padding: 5px;margin:5px;border: green solid;}</style><section id="contenitore"><div>A</div><div>B</div><div>C</div></section>}
  \htmlDual[-s A7][][trim={0 8cm 0 0},clip][css]{<style>}{
#contenitore{
    display: flex;
    border: red solid;
    justify-content: flex-end;
}
}{#contenitore div{padding: 5px;margin:5px;border: green solid;}</style><section id="contenitore"><div>A</div><div>B</div><div>C</div></section>}
\end{frame}

\begin{frame}[fragile]{Allineamento principale con distanziamento}\transfade\centering
  \htmlDual[-s A7][][trim={0 8cm 0 0},clip][css]{<style>}{
#contenitore{
    display: flex;
    border: red solid;
    justify-content: space-around;
}
}{#contenitore div{padding: 5px;margin:5px;border: green solid;}</style><section id="contenitore"><div>A</div><div>B</div><div>C</div></section>}
  \htmlDual[-s A7][][trim={0 8cm 0 0},clip][css]{<style>}{
#contenitore{
    display: flex;
    border: red solid;
    justify-content: space-between;
}
}{#contenitore div{padding: 5px;margin:5px;border: green solid;}</style><section id="contenitore"><div>A</div><div>B</div><div>C</div></section>}

\htmlDual[-s A7][][trim={0 5cm 0 0},clip][css]{<style>}{
  #contenitore{
      display: flex;
      border: red solid;
      height: 200px;
      flex-direction: column;
      justify-content: space-around;
  }
  }{#contenitore div{padding: 5px;margin:5px;border: green solid;}</style><section id="contenitore"><div>A</div><div>B</div><div>C</div></section>}
\end{frame}

\begin{frame}[fragile]{Allineamento secondario}\transfade\centering
  \htmlDual[-s A7][][trim={0 8cm 0 0},clip][css]{<style>}{
#contenitore{
    display: flex;
    border: red solid;
    height: 70px;
}
}{#contenitore div{padding: 5px;margin:5px;border: green solid;}</style><section id="contenitore"><div>A</div><div>B</div><div>C</div></section>}
  \htmlDual[-s A7][][trim={0 8cm 0 0},clip][css]{<style>}{
#contenitore{
    display: flex;
    height: 70px;
    align-items: center;
}
}{#contenitore div{padding: 5px;margin:5px;border: green solid;}</style><section style="border: red solid" id="contenitore"><div>A</div><div>B</div><div>C</div></section>}
\htmlDual[-s A7][][trim={0 8cm 0 0},clip][css]{<style>}{
#contenitore{
  display: flex;
  height: 70px;
  align-items: flex-start;
}
}{#contenitore div{padding: 5px;margin:5px;border: green solid;}</style><section style="border: red solid" id="contenitore"><div>A</div><div>B</div><div>C</div></section>}
\htmlDual[-s A7][][trim={0 8cm 0 0},clip][css]{<style>}{
#contenitore{
  display: flex;
  height: 70px;
  align-items: flex-end;
}
}{#contenitore div{padding: 5px;margin:5px;border: green solid;}</style><section style="border: red solid" id="contenitore"><div>A</div><div>B</div><div>C</div></section>}
\end{frame}


%%%%%% MEMORY
\begin{frame}[fragile]\transfade
  \begin{exercise}\centering
    \begin{columns}
      \begin{column}{.45\textwidth}
        \htmlRender[-s A5][height=.85\textheight, trim={1cm 7cm 1cm 0},clip]{<meta charset="utf-8"><style>img{width:70px !important;height: 70px !important}</style>
    <base href="../../memory/html/">
<style>
body{
    background-color: bisque;
    text-align: center;
}

h1{
    font-family: "Comic Sans MS", cursive, sans-serif;
    font-size: 40pt;
    color: blue;
}

h1, h2, h3{
    margin: 0;
}

.corretto{
    color: green;
}
.errore{
    color: red;
}

#gioco{
    display: flex;
    justify-content: center;
    align-items: center;
}


#tavolo{
    background-color: white;
    border: solid 1px black;
    border-radius: 20px;
    margin: 20px;
    padding: 15px;
}

#gioco aside{
    margin: 20px;
}

#tavolo img{
    height: 100px;
    width: 100px;
}

footer ul li{
    display: inline;
    margin: 0 20px;
}
footer ul a, #gioco button{
    background-color: brown;
    color: white;
    text-align: center;
    text-decoration: none;
    border: none;
    border-radius: 20px;
    padding: 10px 20px;
}
</style>
<header id="top">
    <hgroup>
        <h1> Memory </h1>
        <h2> Fai la tua mossa </h2>
    </hgroup>
</header>

<article id="gioco">
<table id="tavolo">
      <tr>
          <td><img src="img/dorso.png" alt="Dorso"></td>
          <td><img src="img/dorso.png" alt="Dorso"></td>
          <td><img src="img/dorso.png" alt="Dorso"></td>
          <td><img src="img/dorso.png" alt="Dorso"></td>
      </tr>
      <tr>
          <td><img src="img/dorso.png" alt="Dorso"></td>
          <td><img src="img/dorso.png" alt="Dorso"></td>
          <td><img src="img/dorso.png" alt="Dorso"></td>
          <td><img src="img/dorso.png" alt="Dorso"></td>
      </tr>
      <tr>
          <td><img src="img/dorso.png" alt="Dorso"></td>
          <td><img src="img/dorso.png" alt="Dorso"></td>
          <td><img src="img/dorso.png" alt="Dorso"></td>
          <td><img src="img/dorso.png" alt="Dorso"></td>
      </tr>
      <tr>
          <td><img src="img/dorso.png" alt="Dorso"></td>
          <td><img src="img/dorso.png" alt="Dorso"></td>
          <td><img src="img/dorso.png" alt="Dorso"></td>
          <td><img src="img/dorso.png" alt="Dorso"></td>
      </tr>
  </table>

<aside>
    <section>
        <h3 class="corretto"> Giusto </h3>
        <h3 class="errore"> Sbagliato </h3>
        <p> Tentativi: 0 </p>
        <p> Giuste: 0 </p>
    </section>

    <button> Ricomincia </button>
</aside>

</article>

<footer>
    <ul>
        <li> <a href="https://it.wikipedia.org/wiki/Memoria_(gioco)" target="_blank"> Informazioni sul gioco </a>
        <li> <a href="#top"> Torna su </a>
    </ul>
</footer>}
      \end{column}
      \begin{column}{.45\textwidth}
        \begin{enumerate}
          \item Link nel footer \emph{inline} e con \texttt{20px} di margine laterale
          \item Raggruppare le info sul gioco in previsione di una "barra laterale", con \texttt{20px} di margine.
          \item Layout flex per il gioco, centrando il contenuto.
        \end{enumerate}
      \end{column}
    \end{columns}
  \end{exercise}
\end{frame}

\begin{frame}[fragile]\transfade
  \begin{sol}\centering
    \begin{enumerate}
      \item Link nel footer \emph{inline} e con \texttt{20px} di margine laterale
\makeatletter
      \inputminted[breaklines, firstline=50, lastline=53]{css}{\html@dir/\jobname _\thehtml@count.html}
      \makeatother
    \end{enumerate}
  \end{sol}
\end{frame}

\begin{frame}[fragile]\transfade
  \begin{sol}\centering
    \begin{enumerate}
      \setcounter{enumi}{1}
      \item Raggruppare le info sul gioco in previsione di una "barra laterale", con \texttt{20px} di margine.\\
      \texttt{memory.html}:
      \makeatletter
      \inputminted[breaklines, firstline=99, lastline=108]{html}{\html@dir/\jobname _\thehtml@count.html}
      \texttt{memory.css}:
      \inputminted[breaklines, firstline=41, lastline=43]{css}{\html@dir/\jobname _\thehtml@count.html}
      \makeatother
    \end{enumerate}
  \end{sol}
\end{frame}

\begin{frame}[fragile]\transfade
  \begin{sol}\centering
    \begin{enumerate}
      \setcounter{enumi}{2}
      \item Layout flex per il gioco, centrando il contenuto.
      \makeatletter
      \inputminted[breaklines, firstline=26, lastline=30]{html}{\html@dir/\jobname _\thehtml@count.html}
      \makeatother
    \end{enumerate}
  \end{sol}
\end{frame}