% !TeX root = ../../Parte1.tex
\secmeme{html/headbody}
\section{Primi passi}

\begin{frame}[fragile]{Esempio minimale}\transfade\centering
  \begin{minted}{html}
<!DOCTYPE html>
<html lang="it">
    <head>
        <meta charset="utf-8">
        <title>Titolo</title>
    </head>
    <body>
        <!-- Corpo del documento -->
    </body>
</html>
  \end{minted}
\end{frame}

\begin{frame}[fragile]{Metadati}\transfade\centering
  \begin{minted}{html}
<head>
    <meta charset="utf-8">

    <meta name="description" content="Descrizione della pagina">
    <meta name="keywords" content="HTML, HTML5, esempio">
    <meta name="author" content="Autore della pagina">

    <!-- Per dispositivi mobili -->
    <meta name="viewport" content="width=device-width, initial-scale=1.0">
</head>
  \end{minted}
\end{frame}

\begin{frame}[fragile]{Corpo del documento}\transfade\centering
  \htmlDual[-s A7][][trim={0 7cm 0 0},clip]{
<!DOCTYPE html>
<html lang="it">
  <head>
    <title>Titolo</title>
    <meta charset="utf-8">
  </head>
  }{
<body>
    <h1>Titolo</h1>
<!-- h1, h2, h3, h4, h5, h6 -->
    <p>
        Questo è un paragrafo di esempio<br>
        su due righe.
    </p>
</body>
  }{
</html>
  }
\end{frame}

\begin{frame}[fragile]{Suddivisione in sezioni}\transfade\centering
  \htmlDual[-s A7][][trim={0 5cm 0 0},clip]{
<!DOCTYPE html>
<html lang="it">
  <head>
    <title>Titolo</title>
    <meta charset="utf-8">
  </head>
  }{
<body>
    <section>
        <h1>Sezione 1</h1>
        <p>§ 1.1</p>
    </section>
    <section>
        <h1>Sezione 2</h1>
        <p>§ 2.1</p>
        <p>§ 2.2</p>
    </section>
</body>
  }{
</html>
  }
\end{frame}

\begin{frame}{Contenitori: \texttt{section} vs \texttt{article}}\transfade\centering
  \begin{enumerate}[<+->]
    \item \minttag{section} raggruppa contenuti distinti; divide un contenuto più grande.
    \item \minttag{article} raggruppa contenuti simili; ha senso anche da solo.
  \end{enumerate}
  \visible<3->{\minttag{section} e \minttag{article} possono essserre contenuti uno nell'altro.}
\end{frame}

\begin{frame}[fragile]{Contenitori: \texttt{section} e \texttt{article}}\transfade\centering
  \htmlDual[-s A7]{
    <meta charset="utf-8">
    <style>h2{font-size:1.3em}</style>
  }{
<body>
    <article>
        <h1>Quanto è bello HTML</h1>
        <section>
            <h2>Sommario</h2>
            <p>HTML è stupendo.</p>
        </section>
        <section>
            <h2>Introduzione</h2>
            <p>Bla bla bla</p>
        </section>
        ...
        <section>
            <h2>Citazioni</h2>
            <article>
                <p>HTML è super.</p>
                <p> -- Pinco Pallino</p>
            </article>
            ...
        </section>
    </article>
</body>
  }{}
\end{frame}

\begin{frame}{Altri contenitori}\transfade\centering
  \begin{description}[<+->]
    \itemtt[nav] sezione principale collegamenti.
    \itemtt[aside] conenuto aggiuntivo (barra laterale).
    \itemtt[header] intestazione / introduzione della pagina o di un contenitore.
    \itemtt[hgroup] raggruppa titolo e sottotitolo.
    \itemtt[footer] piè di pagina o conclusione di un contenitore.
    \itemtt[figure] conteniene una figura (immagine, video, codice, tabella, ...).
    \itemtt[figcaption] didascalia della figura.
    \itemtt[span] blocco di testo (utile per applicare CSS).
    \itemtt[div] contenitore generico.
  \end{description}
\end{frame}

\begin{frame}[fragile]{Separatore orizzontale}\transfade\centering
  \htmlDual[-s A7][][trim={0 8cm 0 0},clip]{<meta charset="utf-8">}{
<p> Primo paragrafo </p>
<hr>
<p> Secondo paragrafo </p>
  }{}
\end{frame}


%%%%%% MEMORY
\begin{frame}[fragile]\transfade
  \begin{exercise}\centering
    \htmlRender[-s A7][trim={0 5cm 0 0},clip]{
<!DOCTYPE html>
<html lang="it">
    <head>
        <meta charset="utf-8">
        <title> Memory </title>
    </head>
    <body>

        <header>
            <hgroup>
                <h1> Memory </h1>
                <h2> Fai la tua mossa </h2>
            </hgroup>
        </header>

        <article>

            <!-- Tavolo gioco memory -->

            <section>
                <h3> Giusto / Sbagliato </h3>
                <p> Tentativi: 0 </p>
                <p> Giuste: 0 </p>
            </section>

        </article>

    </body>
</html>}
  \end{exercise}
\end{frame}

\begin{frame}[fragile]\transfade
  \begin{sol}\centering
    \makeatletter
    \inputminted[breaklines,fontsize=\tiny]{html}{\html@dir/\jobname _\thehtml@count.html}
    \makeatother
  \end{sol}
\end{frame}