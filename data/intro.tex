% !TeX root = ../Parte1.tex
\secmeme{html/ends}
\section{Sviluppo web}
\frame{\transfade\centering
  \begin{tabular}{c|c|c}
    & Lato client & Lato server \\
    & Front-end & Back-end\\
    \hline
    Struttura & \visible<2->{HTML} &\\
    Stile & \visible<3->{CSS} &\\
    Interazione & \visible<4->{JavaScript} & \visible<5->{Node.js, Python, PHP, \dots}\\
    Gestione dati & & \visible<6->{DBMS (SQL)}\\
  \end{tabular}
}

\endinput

\frame{\transfade\centering
\frametitle{I vantaggi di \LaTeX}
\begin{itemize}
\item<2-> i documenti hanno un'impaginazione perfetta e risultano piacevoli alla lettura
\item<3-> con un po' di esercizio di possono comporre formule matematiche, schemi a blocchi, circuiti,\dots{} semplicemente
\item<4-> la formattazione non subirà mai modifiche drastiche e risulterà uniforme
\end{itemize}
}

\frame{\transfade\centering
\frametitle{Cosa si può fare con \LaTeX?}
\only<1>{
  \begin{columns}\begin{column}{0.4\textwidth}\centering
  Articoli, Relazioni, Libri, Lettere,~\dots
  \end{column}\begin{column}{0.6\textwidth}
  \includegraphics[height=0.8\textheight]{img/relazione}
  \end{column}\end{columns}
}
\only<2>{
  Circuiti, Schemi, Tabelle, Grafici,~\dots\\
  \includegraphics[width=0.9\textwidth]{img/schemi}
}
\only<3>{
  Formule chimiche\\
  \includegraphics[width=0.9\textwidth]{img/chimica}
}
\only<4>{
  Accordi, Spartiti,~\dots\\
  \begin{figure}
    \captionsetup[subfigure]{labelformat=empty}
    \centering
    \begin{subfigure}[b]{0.4\textwidth}
        \includegraphics[width=\textwidth]{img/accordi}
    \end{subfigure}~
    \begin{subfigure}[b]{0.6\textwidth}
        \includegraphics[width=\textwidth]{img/spartiti}
    \end{subfigure}
  \end{figure}
}
\only<5>{
  Presentazioni\\~\\
  \fbox{\includegraphics[width=0.7\textwidth,page=1]{img/presentazione}}
}
\only<6>{
  Scrivere facilmente con più alfabeti\\~\\
  \includegraphics[width=\textwidth,page=1]{img/alfabeti}
}
}

\frame{\transfade\centering
\frametitle{Come si utilizza \LaTeX?}
\begin{itemize}
  \item<2->{Editor di testo + linea di comando}
  \item<3->{Editor avanzato (es: Miktex)}
\end{itemize}
}
