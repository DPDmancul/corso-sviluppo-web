% !TeX root = ../../Parte3.tex
\secmeme{js/dom}
\section[DOM]{DOM\\Document Object Model}
\begin{frame}{Ottenere un elemento}\transfade\centering
  \begin{itemize}
    \item\mintjs{document.getElementById("pippo") // Singolo elemento}\\\smallskip
      \minthtml[fontsize=\scriptsize]{<span id="pippo"></span>}\bigskip
    \item\mintjs{document.getElementsByClassName("pluto") // Array di elementi}\\\smallskip
      \minthtml[fontsize=\scriptsize]{<span class="pluto"></span>}, \minthtml[fontsize=\scriptsize]{<a class="pluto"></a>}\bigskip
    \item\mintjs{document.getElementsByName("paperino") // Array di elementi}\\\smallskip
      \minthtml[fontsize=\scriptsize]{<input name="paperino"></input>}, \minthtml[fontsize=\scriptsize]{<input name="paperino"></input>}\bigskip
    \item\mintjs{document.getElementsByTagName("p") // Array di elementi}\\\smallskip
      \minthtml[fontsize=\scriptsize]{<p></p>}, \minthtml[fontsize=\scriptsize]{<p id="topolino"></p>}
  \end{itemize}
\end{frame}

\begin{frame}[fragile]{Leggere e modificare il contenuto}\transfade\centering
  Leggere il contenuto:
  \begin{minted}[fontsize=\normalfont]{js}
    > let p = document.getElementById("testo");
    > console.log( p.innerHTML );
      "testo originale"
  \end{minted}
  \pause\bigskip
  Modificare il contenuto:
  \begin{minted}[fontsize=\normalfont]{js}
      let p = document.getElementById("testo");
      p.innerHTML = "Testo modificato";
  \end{minted}
\end{frame}

\begin{frame}[fragile]{Leggere e modificare gli attributi}\transfade\centering
  Leggere un attributo:
  \begin{minted}[fontsize=\normalfont]{js}
    > let a = document.getElementById("ancora");
    > console.log( a.href );
      "http://duckduckgo.com"
  \end{minted}
  \pause\bigskip
  Modificare un attributo:
  \begin{minted}[fontsize=\normalfont]{js}
      let a = document.getElementById("ancora");
      a.setAttribute("href", "https://duckduckgo.com");
  \end{minted}
\end{frame}

\begin{frame}[fragile]{Cambiare la classe di un elemento}\transfade\centering
  \begin{minted}[fontsize=\normalfont]{js}
    let pluto = document.getElementById("Pluto");

    // Aggiungi una nuova classe all'elemento
    pluto.classList.add("Pippo");

    // Rimuovi una classe dall'elemento
    pluto.classList.remove("Paperino");
  \end{minted}
\end{frame}


\begin{frame}[fragile]{Creare un nuovo elemento}\transfade\centering
  \begin{minted}[fontsize=\normalfont]{js}
    let p = document.createElement("p"); // Creo un paragrafo

    p.innerHTML = "Testo"; // Inserisco il testo nel paragrafo
    p.classList.add("testo-bello"); // Aggiungo una classe
    p.setAttribute("id", "testo-js"); // Imposto l'id

    let contenitore = document.getElementById("contenitore");

    contenitore.appendChild(p); /* Inserisco il paragrafo
                                   nel contenitore */
  \end{minted}
  \bigskip
  \begin{tabular}{m{.3\textwidth} m{.1\textwidth} m{.4\textwidth}}
    \minthtml[fontsize=\scriptsize]{<div id="contenitore"></div>} &$\to$&\hspace*{-1cm}
    \begin{minipage}{\columnwidth}
      \begin{minted}{html}
<div id="contenitore">
    <p class="testo-bello" id="testo-js">
        Testo
    </p>
</div>
      \end{minted}
    \end{minipage}
  \end{tabular}
\end{frame}


%%%%%% MEMORY
\begin{frame}\transfade
  \begin{exercise}\centering
    Modificare la funzione \mintjs{crea_tavolo} di modo che crei anche il tavolo nella pagina HTML e restituisca anziché \texttt{[i,j]} l'elemento \texttt{img} corrispondente.
  \end{exercise}
\end{frame}

\begin{frame}[fragile]\transfade
  \begin{sol}\centering
      \begin{multicols}{2}
\begin{minted}[fontsize=\tiny]{js}
/**
 * @typedef {{
 *    immagine: HTMLImageElement,
 *    numero: number,
 *    selezionabile: boolean
 * }} Tessera
 */

/**
 * Crea il tavolo da gioco nella pagina HTML.
 * @param {number} size - Dimensione del tavolo da gioco.
 * @return {Tessera[]} Contiene le tessere del tavolo da gioco.
 */
function crea_tavolo(size){

  const board = genera_numeri_tessere(size);

  let tavolo_html = document.getElementById('tavolo');

  /**
   * Tessere del tavolo da gioco
   * @type Tessera[]
   */
  let tavolo = [];

  // Svuota il tavoo da gioco
  tavolo_html.innerHTML = "";

  for(let i = 0; i < size; ++i){
      let riga = document.createElement("tr");
      for (let j = 0; j < size; ++j){
          let cella = document.createElement("td");
          let immagine = document.createElement("img");

          immagine.setAttribute("src", "img/dorso.png");
          immagine.setAttribute("alt", "Dorso");

          tavolo.push({
              immagine: immagine,
              numero: board[ i * size + j ],
              selezionabile: true
          });

          cella.appendChild(immagine);
          riga.appendChild(cella);
      }
      tavolo_html.appendChild(riga);
  }

  return tavolo;
}
      \end{minted}
      \end{multicols}
  \end{sol}
\end{frame}


\begin{frame}\transfade
  \begin{exercise}\centering
    Creare una funzione \mintjs{gestisci_click(elemento, tessere, info, num_coppie)} dove
    \begin{itemize}
      \item \texttt{elemento} è l'elemento \texttt{img} cliccato
      \item \texttt{tessere} è un array contenente le tessere attualmente scoperte (esclusa quella girata e quelle già indovinate)
      \item \texttt{info} è un oggetto \mintjs{{tenativi, giuste}}
      \item \texttt{num\_coppie} il numero totale di coppie di tessere
    \end{itemize}
    La funzione deve
    \begin{enumerate}
      \item "girare" la tessere nell'HTML
      \item se già due tessere sono state scoperte:
        \begin{enumerate}
          \item incrementare il numero di tentativi
          \item verificare se è stata fatta una coppia (e se tutte le tessere sono state abbinate congratularsi col giocatore)
          \item altrimenti rigirarle dopo qualche secondo (utilizzare \mintjs{setTimeout(callback, millisec)})
        \end{enumerate}
    \end{enumerate}
  \end{exercise}
\end{frame}


\begin{frame}[fragile]\transfade
  \begin{sol}\centering
    memory.css
      \begin{minted}{css}
.nascosto{
    display: none;
}
      \end{minted}
      memory.html
      \begin{itemize}
        \item Giusto/Sbagliato inizialmente nascosto
        \item Il numero di tentativi e di coppie giuste inserito in uno span per poterli modificare da JS
      \end{itemize}
      \begin{minted}{html}
<h3 id="corretto" class="corretto nascosto"> Giusto </h3>
<h3 id="errore" class="errore nascosto"> Sbagliato </h3>
<p> Tentativi: <span id="tentativi">0</span> </p>
<p> Giuste: <span id="giuste">0</span> </p>
      \end{minted}
  \end{sol}
\end{frame}
\begin{frame}[fragile]\transfade
  \begin{sol}\centering
    memory.js
\begin{minted}[fontsize=\tiny]{js}
/**
 * @typedef {{
 *    tentativi: number,
 *    giuste: number
 * }} Info
 */

/**
 * Gestisce l'evento click su un'`img` selezionabile del tavolo.
 * @param {Tessera} elemento - Elemento su cui è avvenuto il click.
 * @param {Tessera[]} tessere - Contiene gli elementi delle tessere precedentemente scoperte.
 * @param {Info} info - Contiene il numero di tenativi e di abbinamenti corretti.
 * @param {number} num_coppie - Numero di coppie totali.
 */
function gestisci_click(elemento, tessere, info, num_coppie){
    if(tessere.length === 2) // Ci sono già due tessere girate
        return;

    tessere.push(elemento);
    elemento.selezionabile = false;
    elemento.immagine.setAttribute("src", `img/${elemento.numero}.png`);
    elemento.immagine.setAttribute("alt", DESCRIZIONE[elemento.numero]);

      \end{minted}
      \dots
  \end{sol}
\end{frame}
\begin{frame}[fragile]\transfade
  \begin{sol}\centering\dots
\begin{minted}[fontsize=\tiny]{js}
    if(tessere.length === 2){ // È la seconda tessera girata
        document.getElementById("tentativi").innerText = ++info.tentativi;

        if(tessere[0].numero === tessere[1].numero){ // Corretto
            document.getElementById("errore").classList.add("nascosto");
            document.getElementById("corretto").classList.remove("nascosto");
            document.getElementById("giuste").innerText = ++info.giuste;

            // Svuota l'array delle tessere attuali
            tessere.length = 0

            if (info.giuste === num_coppie){
                alert("Complimenti!");
                inizio();
            }
        }else{ // Errore
            document.getElementById("corretto").classList.add("nascosto");
            document.getElementById("errore").classList.remove("nascosto");

            // Richiudi le tessere dopo 1 secondo
            setTimeout( () =>{
                for(let t of tessere){
                    t.immagine.setAttribute("src", "img/dorso.png");
                    t.immagine.setAttribute("alt", "Dorso");
                    t.selezionabile = true;
                }
                // Svuota l'array delle tessere attuali
                tessere.length = 0
            }, 1000 /* ms */);
        }
    }
}
      \end{minted}
  \end{sol}
\end{frame}