% !TeX root = ../../Parte3.tex
\secmeme{js/dom}
\section[DOM]{DOM\\Document Object Model}
\begin{frame}{Ottenere un elemento}\transfade\centering
  \begin{itemize}
    \item\mintjs{document.getElementById("pippo") // Singolo elemento}\\\smallskip
      \minthtml[fontsize=\scriptsize]{<span id="pippo"></span>}\bigskip
    \item\mintjs{document.getElementsByClassName("pluto") // Array di elementi}\\\smallskip
      \minthtml[fontsize=\scriptsize]{<span class="pluto"></span>}, \minthtml[fontsize=\scriptsize]{<a class="pluto"></a>}\bigskip
    \item\mintjs{document.getElementsByName("paperino") // Array di elementi}\\\smallskip
      \minthtml[fontsize=\scriptsize]{<input name="paperino"></input>}, \minthtml[fontsize=\scriptsize]{<input name="paperino"></input>}\bigskip
    \item\mintjs{document.getElementsByTagName("p") // Array di elementi}\\\smallskip
      \minthtml[fontsize=\scriptsize]{<p></p>}, \minthtml[fontsize=\scriptsize]{<p id="topolino"></p>}
  \end{itemize}
\end{frame}

\begin{frame}[fragile]{Leggere e modificare il contenuto}\transfade\centering
  Leggere il contenuto:
  \begin{minted}[fontsize=\normalfont]{js}
    > let p = document.getElementById("testo");
    > console.log( p.innerHTML );
      "testo originale"
  \end{minted}
  \pause\bigskip
  Modificare il contenuto:
  \begin{minted}[fontsize=\normalfont]{js}
      let p = document.getElementById("testo");
      p.innerHTML = "Testo modificato";
  \end{minted}
\end{frame}

\begin{frame}[fragile]{Leggere e modificare gli attributi}\transfade\centering
  Leggere un attributo:
  \begin{minted}[fontsize=\normalfont]{js}
    > let a = document.getElementById("ancora");
    > console.log( a.href );
      "http://duckduckgo.com"
  \end{minted}
  \pause\bigskip
  Modificare un attributo:
  \begin{minted}[fontsize=\normalfont]{js}
      let a = document.getElementById("ancora");
      a.setAttribute("href", "https://duckduckgo.com");
  \end{minted}
\end{frame}

\begin{frame}[fragile]{Cambiare la classe di un elemento}\transfade\centering
  \begin{minted}[fontsize=\normalfont]{js}
    let pluto = document.getElementById("Pluto");

    // Aggiungi una nuova classe all'elemento
    pluto.classList.add("Pippo");

    // Rimuovi una classe dall'elemento
    pluto.classList.remove("Paperino");
  \end{minted}
\end{frame}


\begin{frame}[fragile]{Creare un nuovo elemento}\transfade\centering
  \begin{minted}[fontsize=\normalfont]{js}
    let p = document.createElement("p"); // Creo un paragrafo

    p.innerHTML = "Testo"; // Inserisco il testo nel paragrafo
    p.classList.add("testo-bello"); // Aggiungo una classe
    p.setAttribute("id", "testo-js"); // Imposto l'id

    let contenitore = document.getElementById("contenitore");

    contenitore.appendChild(p); /* Inserisco il paragrafo
                                   nel contenitore */
  \end{minted}
  \bigskip
  \begin{tabular}{m{.3\textwidth} m{.1\textwidth} m{.4\textwidth}}
    \minthtml[fontsize=\scriptsize]{<div id="contenitore"></div>} &$\to$&\hspace*{-1cm}
    \begin{minipage}{\columnwidth}
      \begin{minted}{html}
<div id="contenitore">
    <p class="testo-bello" id="testo-js">
        Testo
    </p>
</div>
      \end{minted}
    \end{minipage}
  \end{tabular}
\end{frame}