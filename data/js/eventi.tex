% !TeX root = ../../Parte3.tex
\secmeme{js/event}
\section[Eventi]{Eventi HTML}
\begin{frame}{Tipi di evento}\transfade\centering
  \begin{description}
    \item[onclick] Al click sull'elemento HTML.
    \item[onmouseover] Quando il mouse passa sopra l'elemento.
    \item[onmouseout] Quando il mouse si sposta dall'elemento.
    \item[onkeydown] Quando viene premuto un tasto della tastiera (si usa negli \minttag{input}).
    \item[onload] Al caricamento dell'elemento (si usa nel \minttag{body}).
    \item[onchange] Quando un elemento viene cambiato (si usa negli \minttag{input}).
  \end{description}
\end{frame}

\begin{frame}[fragile]{Registrare un evento}\transfade\centering
  \begin{itemize}[<+->]
    \item Da HTML\\
      \minthtml{<button onclick="alert('Mi hai premuto!')">Premimi</button> }
    \item Da JS\\
      \minthtml{<button id="pulsante">Premimi</button> }\\\dots\\
      \begin{minted}[breaklines]{js}
document.getElementById("pulsante").onclick = () => alert("Mi hai premuto!");
      \end{minted}
  \end{itemize}
\end{frame}




%%%%%% MEMORY
\begin{frame}\transfade
  \begin{exercise}\centering
    Creare una funzione \texttt{inizio} (invocata al caricamento della pagina e alla pressione del bottone \emph{Ricomincia}) che
    \begin{enumerate}
      \item crea un tavolo usando la funzione \texttt{crea\_tavolo}
      \item inizializza gli elementi HTML di feedback (corretto, giusto, tentativi, giuste)
      \item crea un vettore vuoto per contenere le celle attualmente aperte
      \item un oggetto \texttt{info} con tentativi e giuste a zero
      \item per ogni immagine del tavolo registra all'evento di click col mouse l'esecuzione di \texttt{gestione\_click}
    \end{enumerate}
  \end{exercise}
\end{frame}

\begin{frame}[fragile]\transfade
  \begin{sol}\centering
    memory.html
    \begin{minted}[breaklines]{html}
      <body onload="inizio()">
      ...
      <button onclick="inizio()"> Ricomincia </button>
    \end{minted}
  \end{sol}
\end{frame}

\begin{frame}[fragile]\transfade
  \begin{sol}\centering
    memory.js
    \begin{minted}[breaklines]{js}
/**
 * Avvia una nuova partita.
 */
function inizio(){
    // Crea e disegna tavolo
    let tavolo = crea_tavolo(SIZE);

    // Inizializza elementi HTML
    document.getElementById("corretto").classList.add("nascosto");
    document.getElementById("errore").classList.add("nascosto");
    document.getElementById("tentativi").innerText = 0;
    document.getElementById("giuste").innerText = 0;

    /**
     * Contiene le tessere girate in questa mano di gioco.
     * @type Tessera[]
     */
    let tessere_attuali = [];

    \end{minted}
    \dots
  \end{sol}
\end{frame}
\begin{frame}[fragile]\transfade
  \begin{sol}\centering
    \dots
    \begin{minted}[breaklines]{js}
    /**
     * Contiene le informazioni sulla partita.
     * @type Info
     */
    let info = {
        tentativi: 0,
        giuste: 0
    };

    // Gestione evento click
    for(let e of tavolo)
        e.immagine.onclick = () => {
            if(e.selezionabile)
                gestisci_click(e, tessere_attuali, info, SIZE*SIZE/2);
        };
}
    \end{minted}
  \end{sol}
\end{frame}