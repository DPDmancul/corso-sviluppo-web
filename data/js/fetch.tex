% !TeX root = ../../Parte3.tex
\secmeme{js/async}
\section{Fetch API}
\begin{frame}{Richieste HTTP}\transfade\centering
  Per interagire con un backend (es: inviare dati form, caricare nuovi dati nella pagina, \dots) si usano le richieste HTTP.
  \bigskip\\\pause
  \alert{Attenzione!} L'url della pagina e della richiesta devono essere sullo stesso dominio oppure l'url della pagina che fa la richiesta deve essere accettato dal server.
\end{frame}

\begin{frame}{Metodi}\transfade\centering
  \begin{description}
    \item[GET] Richiesta dati (dati nell'url: non usare per dati sensibili)\\\pause
      \texttt{https://duckduckgo.com/\textbf{?q=fetch+api}} $\implies$ OK\\\pause
      \texttt{https://google.it/login/\textbf{?user=admin\&pass=1234}} $\implies$ NO!!!\pause
    \item[POST] Invio dati (i dati sono al sicuro nel corpo della richiesta e non visibili nell'url)\pause
    \item[PUT]
    \item[DELETE]
    \item[\dots]
  \end{description}
\end{frame}

\begin{frame}[fragile]{Richiesta GET semplice}\transfade\centering
  \begin{minted}{js}
fetch("https://duckduckgo.com/")    // richiesta GET
.then(risposta => risposta.text())  // quando arriva la risposta estrai il contenuto
.then(testo => console.log(testo)) /* callback
                              (esegue le operazioni col contenuto della risposta) */
.catch(err => {
    // eventuale gestione dell'errore
});
  \end{minted}
\bigskip\pause
  o in alternativa senza callback:
  \begin{minted}{js}
let testo = await fetch("https://duckduckgo.com/")
                  .then(risposta => risposta.text());
console.log(testo);
  \end{minted}
  \smallskip
  \texttt{await} blocca l'esecuzione del codice in attesa della risposta.\\\pause
  \texttt{await} può essere usato solo dentro funzioni asincrone:
  \begin{minted}{js}
    async function get(url){
        let testo = await fetch("https://duckduckgo.com/")
                        .then(risposta => risposta.text());
        console.log(testo);
    }
  \end{minted}
\end{frame}

\begin{frame}[fragile]{Richiesta GET di JSON}\transfade\centering
  JSON (JavaScript Object Notation) è un formato per scambiare facilmente oggetti javascript fra client e server.
  \smallskip
  \begin{minted}{json}
    {
      "lista" : [1, 2, 3, 4],
      "numero": 12.4,
      "testo" : "Lorem ipsum dolor sit amen"
    }
  \end{minted}
  \bigskip\pause
  \begin{minted}{js}
fetch("https://gitlab.com/groups/cominfo/-/children.json") // richiesta GET
.then(risposta => risposta.json()) // quando arriva la risposta estrai l'oggetto
.then(obj => console.log(obj)); // callback
                              // (esegue le operazioni con l'oggetto della risposta)
  \end{minted}
\end{frame}

\begin{frame}[fragile]{Richiesta POST semplice}\transfade\centering
  \begin{minted}{js}
// Preapra i dati per l'invio
const dati = new FormData();
dati.append("q", "fetch api");

fetch("https://duckduckgo.com/", {
    method: "POST", // la richiesta sarà POST e non GET
    body: dati, // Invia i dati
})
// segue come con risposta GET:
.then(rispsota => rispsota.text())
.then(testo => console.log(testo));
  \end{minted}
\end{frame}

\begin{frame}[fragile]{Richiesta POST con JSON}\transfade\centering
  \begin{minted}{js}
const dati = {content : "test:\n  script: exit"};

fetch("https://gitlab.com/api/v4/ci/lint", {
    method: "POST", // la richiesta sarà POST e non GET
    headers: {"Content-Type" : 'application/json'}, /* Avvisa il server che
                                                  stiamo inviando un JSON*/
    body: JSON.stringify(dati), // Crea un JSON dell'oggetto `dati`
})
// segue come con risposta GET:
.then(rispsota => rispsota.json())
.then(obj => console.log(obj));
  \end{minted}
\end{frame}