% !TeX root = ../../Parte3.tex
\secmeme{js/equal}
\section{Strutture di controllo}
\begin{frame}[fragile]{Operatori di comparazione}\transfade\centering
  \begin{description}
    \item[\mintjs{==}] Uguaglianza (stesso valore)
    \item[\mintjs{===}] Uguaglianza stretta (anche stesso tipo)\smallskip
\begin{minted}{js}
> 1 == "1"
  true
> 1 === "1"
  false
\end{minted}
    \pause\smallskip
    \item[\mintjs{!=}] Disuguaglianza
    \item[\mintjs{!==}] Disuguaglianza stretta
    \pause\smallskip
    \item[\mintjs{>}] Maggiore
    \item[\mintjs{>=}] Maggiore o uguale
    \item[\mintjs{<}] Minore
    \item[\mintjs{<=}] Minore o uguale
  \end{description}
\end{frame}

\begin{frame}[fragile]{Operatori logici}\transfade\centering
  \begin{description}[<+->]
    \item[\mintjs{&&}] Congiunzione logica (\emph{et}, \emph{and}, $\land$)\\\smallskip
\begin{minipage}{.3\linewidth}
\begin{minted}{js}
> false && false
  false
> false && true
  false
\end{minted}
\end{minipage}
\begin{minipage}{.3\linewidth}
\begin{minted}{js}
> true && false
  false
> true && true
  true
\end{minted}
\end{minipage}\smallskip
    \item[\mintjs{||}] Disgiunzione inclusiva (\emph{vel}, \emph{or}, $\lor$)\\\smallskip
\begin{minipage}{.3\linewidth}
\begin{minted}{js}
> false || false
  false
> false || true
  true
\end{minted}
\end{minipage}
\begin{minipage}{.3\linewidth}
\begin{minted}{js}
> true || false
  true
> true || true
  true
\end{minted}
\end{minipage}\smallskip
    \item[\mintjs{!}] Negazione logica o complemento logico (\emph{not}, $\lnot$)\smallskip
\begin{minted}{js}
> !false
  true
> !true
  false
\end{minted}
  \end{description}
\end{frame}

\begin{frame}[fragile]{If}\transfade\centering
  \begin{minted}[fontsize=\normalsize]{js}
    if ( /* condizione */ ) {
        // codice eseguito solo se la condizine è vera
    } else {
        // codice eseguito solo se la condizine è falsa
    }
  \end{minted}
  \pause\bigskip
  La clausola \mintjs{else{}} è opzionale.
  \begin{minted}[fontsize=\normalsize]{js}
    if ( a%2 == 0 )
        console.log(`${a} è pari`);
  \end{minted}
\end{frame}

\begin{frame}[fragile]{Switch}\transfade\centering
  \begin{columns}
    \begin{column}{.5\textwidth}
      \begin{minted}[fontsize=\normalsize]{js}
if ( num === 0 )
    console.log("zero");
else if ( num === 1 )
    console.log("uno");
else if ( num === 2 )
    console.log("due");
else
    console.log("altro");
      \end{minted}
    \end{column}
    \pause
    \begin{column}{.5\textwidth}
      \begin{minted}[fontsize=\normalsize]{js}
switch ( num ) {
    case 0:
        console.log("zero");
        break;
    case 1:
        console.log("uno");
        break;
    case 2:
        console.log("due");
        break;
    default:
        console.log("altro");
}
      \end{minted}
    \end{column}
  \end{columns}
\end{frame}

\begin{frame}[fragile]{While e do-while}\transfade\centering
  \begin{minted}[fontsize=\normalsize]{js}
    while ( /* condizione */ ) {
        // codice eseguito a ripetizione
        // finché la condizione è vera
    }
  \end{minted}
  Se la condizione inizialmente è falsa il codice non viene mai eseguito.
  \pause\bigskip
  \begin{minted}[fontsize=\normalsize, breaklines]{js}
    do {
        // codice eseguito a ripetizione
        // finché la condizione è vera,
        // ma comuqnue almeno una volta
    } while ( /* condizione */ );
  \end{minted}
\end{frame}

\begin{frame}[fragile]{For}\transfade\centering
  \begin{columns}
    \begin{column}{.5\textwidth}
      \begin{minted}[fontsize=\normalsize]{js}
let a = [1,2,3];
let i = 0;
while(i < a.length){
    console.log(a[i]);
    ++i; // equivale a i+=1
         //          e i=i+1
}
      \end{minted}
    \end{column}
    \pause
    \begin{column}{.5\textwidth}
      \begin{minted}[fontsize=\normalsize]{js}
let a = [1,2,3];
for(let i=0; i < a.length; ++i)
    console.log(a[i]);
      \end{minted}
    \end{column}
  \end{columns}
\end{frame}

\begin{frame}[fragile]{For of}\transfade\centering
\begin{minted}[fontsize=\normalsize]{js}
    let a = [1,2,3];
    for(const e of a)
        console.log(e);
\end{minted}
\end{frame}


%%%%%% MEMORY
\begin{frame}[fragile]\transfade
  \begin{exercise}\centering
    \begin{enumerate}
      \item Riempire il tavolo da gioco (\texttt{board}) con coppie di numeri $0,0,1,1,2,2,\dots$.
      \item Mescolare il tavolo da gioco con
      \begin{algorithm}[H]\normalfont\small
        \TitleOfAlgo{Fisher-Yates, variante di Knuth}
        \KwIn{Vettore V di N elementi da mescolare}
        \KwOut{Vettore V mescolato}
        \For{$i\leftarrow N-1$ \KwTo $0$}{
          j $\leftarrow$ intero random $\colon$ 0 $\leqslant$ j < i \tcp*{ovvero $\lfloor$ random($[0,1)$)$\cdot$ i $\rfloor$~~\\in JS \mintjs{Math.floor(Math.random() * i})~~}
          [V[i], V[j]] = [V[j], V[i]] \tcp*{Scambia elementi in posizione i e j~~}
        }
      \end{algorithm}
    \end{enumerate}
  \end{exercise}
\end{frame}

\begin{frame}[fragile]\transfade
  \begin{sol}\centering
    \begin{enumerate}
      \item Riempire il tavolo da gioco (\texttt{board}) con coppie di numeri $0,0,1,1,2,2,\dots$.
      \begin{minted}{js}
        // Riempie il tavolo da gioco
        for(let i = 0; i < SIZE*SIZE; ++i)
            // Dividendo per due otteniamo coppie di numeri:
            // [0,0,1,1,2,2,...]
            board[i] = Math.floor(i/2);
      \end{minted}
    \item Mescolare il tavolo da gioco con
      \begin{algorithm}[H]\normalfont\small
        \For{$i\leftarrow N-1$ \KwTo $0$}{
          j $\leftarrow \lfloor$ random($[0,1)$)$\cdot$ i $\rfloor$\;
          [V[i], V[j]] = [V[j], V[i]]\;
        }
      \end{algorithm}{\scriptsize
        V = \texttt{board}; N = \texttt{SIZE*SIZE}
      }
      \begin{columns}
        \begin{column}{.4\textwidth}
\begin{minted}[fontsize=\tiny]{js}
// Mescola il tavolo da gioco
// Alogoritmo Fisher-Yates, variante di Knuth
for(let i = SIZE*SIZE-1; i > 0; --i){
    const j = Math.floor(Math.random() * i);
    // Scambia gli elementi in posizione i e j
    [board[i], board[j]] = [board[j], board[i]];
}
\end{minted}
        \end{column}
        \begin{column}{.4\textwidth}
\begin{minted}[fontsize=\tiny]{js}
// Equivalentemente
for(let i = SIZE*SIZE-1; i > 0; --i){
    const j = Math.floor(Math.random() * i);
    let temp = board[i];
    board[i] = board[j];
    board[j] = temp;
}
\end{minted}
        \end{column}
      \end{columns}
    \end{enumerate}
  \end{sol}
\end{frame}