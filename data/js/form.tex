% !TeX root = ../../Parte3.tex
\secmeme{js/validation}
\section[Form]{Gestione dei form}

\begin{frame}[fragile]{Validazione}\transfade\centering
  \begin{minted}{html}
<form name="login" onsubmit="return valida_form()">
  <input type="text" name="user" placeholder="Username">
  <input type="submit" value="Registrati">
</form>

<script>
  function valida_form() {
    const user = document.forms["login"]["user"].value;
    if (user.length < 4) {
      alert("L'username deve avere almeno 4 caratteri.");
      return false;
    }
    return true; // Questo per i vecchi form:
                 // con le fetch API si mette sempre `return false`
  }
</script>
  \end{minted}
  \pause\par
  \alert{Attenzione!} I controlli lato client possono essere facilmente bypassati: servono solo a dare un feedback all'utente e vanno rifatti lato server per garantire la sicurezza del sistema.\\
\end{frame}

\begin{frame}[fragile]{Invio di un form con fetch API}\transfade\centering
  \begin{minted}{html}
<form name="login" onsubmit="return invia_form()">
  <input type="text" name="user" placeholder="Username">
  <input type="submit" value="Registrati">
</form>

<script>
  function invia_form() {
    //eventuale validazione

    const dati = new FormData(document.forms["login"]);

    fetch("https://mio_backedn.local/", {
        method: "POST",
        body: dati, // Invia il form
    })
    .then(rispsota => rispsota.text())
    .then(testo => console.log(testo));

    return false; // Non usare il vecchio sistema di invio form
  }
</script>
  \end{minted}
\end{frame}

\begin{frame}[fragile]{Invio di un form in JSON con fetch API}\transfade\centering
  \begin{minted}{html}
<form name="login" onsubmit="return invia_form()">
  <input type="text" name="user" placeholder="Username">
  <input type="submit" value="Registrati">
</form>

<script>
  function invia_form() {
    //eventuale validazione

    const dati = new FormData(document.forms["login"]); // FormData
    const dati_obj = Object.fromEntries(dati); // Oggetto

    fetch("https://mio_backedn.local", {
        method: "POST",
        headers: {"Content-Type" : 'application/json'},
        body: JSON.stringify(dati_obj),  // Invia il form come JSON
    })
    .then(rispsota => rispsota.json())
    .then(obj => console.log(obj));

    return false; // Non usare il vecchio sistema di invio form
  }
</script>
  \end{minted}
\end{frame}