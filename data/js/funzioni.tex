% !TeX root = ../../Parte3.tex
\secmeme{js/fun}
\section{Funzioni}
\begin{frame}[fragile]{Funzioni}\transfade\centering
\begin{minted}[fontsize=\normalsize]{js}
  function somma (a, b){
      return a + b;
  }

  console.log( somma(1, -.5) ); // 0.5
\end{minted}
\end{frame}

\begin{frame}[fragile]{Parametri mancanti}\transfade\centering
  \begin{minted}[fontsize=\normalsize]{js}
    function somma (a, b){
        return a + b;
    }

    console.log( somma(1) ); // NaN : not a number
  \end{minted}
  \pause\bigskip
  \begin{minted}[fontsize=\normalsize]{js}
    function somma (a, b = 0){
        return a + b;
    }

    console.log( somma(1) ); // 1
  \end{minted}
\end{frame}

\begin{frame}[fragile]{Modifica degli argomenti}\transfade\centering
  \begin{minted}[fontsize=\normalsize]{js}
    function fun (a, b, c){
        a = 0;
        b = [];
        c[0] = 0;
        c.push(4);
    }

    let pippo = 1;
    let pluto = [1,2,3];
    let paperino = [1,2,3];

    fun(pippo, pluto, paperino);

    console.log(pippo);     // 1
    console.log(pluto);     // [1,2,3]
    console.log(paperino);  // [0,2,3,4]
  \end{minted}
  \bigskip
  \begin{itemize}
    \item Se riassegniamo un argomento la modifica avrà effetto solo nel corpo della funzione.
    \item Se operiamo su un argomento la modifica avverrà anche fuori dal corpodella funzione.
  \end{itemize}
\end{frame}

\begin{frame}[fragile]{Funzioni anonime (lambda)}\transfade\centering
  \begin{minted}[fontsize=\normalsize]{js}
    const somma = function (a, b){
        return a + b;
    }
  \end{minted}
  \pause\bigskip
  In alternativa, usando le \emph{funzioni a freccia}:
  \begin{minted}[fontsize=\normalsize]{js}
            const somma = (a, b) => a + b;
// oppure:  const somma = (a, b) => {return a + b};
\end{minted}
\end{frame}