% !TeX root = ../../Parte3.tex
\secmeme{js/js}
\section{JavaScript}
\begin{frame}[fragile]{\sout{Java}ECMAScript}\transfade\centering
  Linguaggio di programmazione (scripting) per pagine web.\\\bigskip\pause
  Orientato agli oggetti (ma senza classi) e agli eventi.\\\bigskip\pause
  Creato "in fretta" da Netscape, ora standardizzato dalla ECMA.\\
\end{frame}

\begin{frame}[fragile]{JS in HTML}\transfade\centering
  Ci sono tre modi di aggiungere uno script JS ad una pagina HTML:
  \begin{enumerate}[<+(1)->]
    \item \textbf{Internal} Mediante codice inserito nel tag \minttag{script}:
    \begin{minted}{html}
<script>
    // Codice JS
</script>
    \end{minted}
    \item \textbf{External} Mediante un file sorgente esterno:
    \begin{minted}{html}
<script src="myscript.js"></script>
    \end{minted}
    \item \textbf{Inline} Per la gestione degli eventi.
    \begin{minted}{html}
<span onclick="/* Codice JS */"> Cliccami </span>
    \end{minted}
  \end{enumerate}
  \visible<5->{Il tag \minttag{script} può troavarsi sia nell'\minttag{head} che nel \minttag{body} (solitamente in fondo). Viene eseguito dove si trova.}\\
\end{frame}