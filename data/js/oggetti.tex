% !TeX root = ../../Parte3.tex
\secmeme{js/js_aspirina}
\section{Oggetti}
\begin{frame}[fragile]{Oggetti}\transfade\centering
  \begin{minted}[fontsize=\normalsize]{js}
    let utente = {

        // Proprietà
        nome: "Franco",
        cognome: "Scaglia",

        // Metodi
        compleanno: () => {
            const oggi = new Date();
            const dd = oggi.getDate();
            const mm = oggi.getMonth() + 1;

            return `${dd}/${mm}`;
        }
    };
  \end{minted}
\end{frame}

\begin{frame}[fragile]{Accedere alle proprietà}\transfade\centering
  \begin{minted}[fontsize=\normalsize]{js}
    > utente.nome
      "Franco"
    > utente['nome']
      "Franco"

    > utente.nome_completo = function () {
          return `${this.nome} ${this.cognome}`;
      }
    > utente.nome_completo()
      "Franco Scaglia"
  \end{minted}
  \bigskip
  \alert{Attenzione!} \mintjs{this} NON è disponibile se dichiariamo il metodo con una funzione a freccia.\\
\end{frame}

\begin{frame}[fragile]{For in}\transfade\centering
  \begin{minted}[fontsize=\normalsize]{js}
    > let obj = { a: 1, b: 'a', c: [1,2,3] };
    > for( p in obj )
          console.log(p, obj[p]);
      "a" 1
      "b" "a"
      "c" [ 1, 2, 3 ]
  \end{minted}
\end{frame}