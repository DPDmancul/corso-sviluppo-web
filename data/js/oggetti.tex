% !TeX root = ../../Parte3.tex
\secmeme{js/js_aspirina}
\section{Oggetti}
\begin{frame}[fragile]{Oggetti}\transfade\centering
  \begin{minted}[fontsize=\normalsize]{js}
    let utente = {

        // Proprietà
        nome: "Franco",
        cognome: "Scaglia",

        // Metodi
        compleanno: () => {
            const oggi = new Date();
            const dd = oggi.getDate();
            const mm = oggi.getMonth() + 1;

            return `${dd}/${mm}`;
        }
    };
  \end{minted}
\end{frame}

\begin{frame}[fragile]{Accedere alle proprietà}\transfade\centering
  \begin{minted}[fontsize=\normalsize]{js}
    > utente.nome;
      "Franco"
    > utente["nome"];
      "Franco"

    > utente.nome_completo = function () {
          return `${this.nome} ${this.cognome}`;
      }
    > utente.nome_completo();
      "Franco Scaglia"
  \end{minted}
  \bigskip
  \alert{Attenzione!} \mintjs{this} NON è disponibile se dichiariamo il metodo con una funzione a freccia.\\
\end{frame}

\begin{frame}[fragile]{For in}\transfade\centering
  \begin{minted}[fontsize=\normalsize]{js}
    > let obj = { a: 1, b: 'a', c: [1,2,3] };
    > for( p in obj )
          console.log(p, obj[p]);
      "a" 1
      "b" "a"
      "c" [ 1, 2, 3 ]
  \end{minted}
\end{frame}



%%%%%% MEMORY
\begin{frame}[fragile]\transfade
  \begin{exercise}\centering
    Creare funzione \mintjs{crea_tavolo(size)} che dato un tavolo \mintjs{board = genera_numeri_tessere(size)} ritorna un vettore di oggetti rappresentanti una tessera:
    \begin{minted}{js}
{
  immagine: [i,j],    // coordinate della tessera nel tavolo
  numero: board[k],   // numero dell'immagine della tessera
  selezionabile: true // la cella è coperta, e quindi selezionabile
}
    \end{minted}
  \end{exercise}
\end{frame}

\begin{frame}[fragile]\transfade
  \begin{sol}\centering
      \begin{minted}[fontsize=\tiny]{js}
/**
 * @typedef {{
 *    immagine: number[],
 *    numero: number,
 *    selezionabile: boolean
 * }} Tessera
 */

/**
 * Crea il tavolo da gioco nella pagina HTML.
 * @param {number} size - Dimensione del tavolo da gioco.
 * @return {Tessera[]} Contiene le tessere del tavolo da gioco.
 */
function crea_tavolo(size){
    const board = genera_numeri_tessere(size);
    /**
     * Tessere del tavolo da gioco
     * @type Tessera[]
     */
    let tavolo = [];
    for(let i = 0; i < size; ++i){
        for (let j = 0; j < size; ++j){
            tavolo.push({
                immagine: [i,j],
                numero: board[ i * size + j ],
                selezionabile: true
            });
        }
    }
    return tavolo;
}
      \end{minted}
  \end{sol}
\end{frame}