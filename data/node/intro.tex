% !TeX root = ../../Parte4.tex
\secmeme{node/modules_anakin}
\section{Node.js}
\begin{frame}{Node.js}\transfade\centering
  Interprete JavaScript\footnote{Lo stesso di Chromium} per eseguire JS lato server.\\\bigskip\pause
  Orientato agli eventi (dorme finché non arriva una richiesta)
  \\\pause
  Massimizza throughput e permette scalabilità.\\
\end{frame}

\begin{frame}[fragile]{Un server semplice}\transfade\centering
  \begin{minted}{js}
"strict mode";

const http = require("http");

const hostname = "127.0.0.1"; // localhost
const port = 3000;

const server = http.createServer((req, res) => {
    res.statusCode = 200; // Tutto è andato a buon fine
    res.setHeader('Content-Type', 'text/plain'); // Tipo di risposta (testo semplice)
    res.end('Hello World!'); // Il corpo della risposta
});

server.listen(port, hostname, () => {
    console.log(`Server running at http://${hostname}:${port}/`);
});
  \end{minted}
  ~\\\bigskip
  Per eseguirlo: \mint{bash}{node server.js}
\end{frame}

\begin{frame}{Node Package Manager (npm)}\transfade\centering
  Permette di gestire facilmente i progetti di Node.js.\\\bigskip\pause
  Ha problemi di "peso": se si lavora a tanti progetti sulla stessa macchina conviene usare \texttt{pnpm}.\\
\end{frame}

\begin{frame}[fragile]{Creiamo un progetto con (p)npm}\transfade\centering
  \begin{enumerate}[<+->]
    \item Creare un file \texttt{package.json} all'interno della cartella del progetto:
    \begin{minted}[fontsize=\normalsize]{json}
      {
          "name": "memory",
          "version": "1.0.0",
          "description": "Semplice gioco del memory",
          "author": "Davide Peressoni"
      }
    \end{minted}
    \item Creare un file \texttt{server.js} col corpo del programma
    \item Eseguire il progetto con \texttt{npm start} / \texttt{pnpm start}
  \end{enumerate}
\end{frame}

\begin{frame}[fragile]{Aggiungere una libreria ad un progetto}\transfade\centering
  \begin{enumerate}[<+->]
    \item Per ogni pacchetto da aggiungere dare:
    \begin{minted}{bash}
       npm add nome_pacchetto
      pnpm add nome_pacchetto
    \end{minted}
    \item[] Il file \texttt{package.json} risulterà aggiornato:
    \begin{minted}{json}
      {
          "name": "memory",
          "version": "1.0.0",
          "description": "Semplice gioco del memory",
          "author": "Davide Peressoni",
          "dependencies": {
              "express": "^4.17.1"
          }
      }
    \end{minted}
    \item Installare i pacchetti aggiunti con
    \begin{minted}{bash}
      npm install
    \end{minted}
    Il pacchetto sarà installato nella cartella \texttt{node\_modules}.
    \makeatletter
    \let\temp@theenumi\theenumi
    \def\theenumi{}
    \makeatother
    \item Per aggiornare i pacchetti installati
    \makeatletter
    \let\theenumi\temp@theenumi
    \makeatother
    \begin{minted}{bash}
      npm update
    \end{minted}
  \end{enumerate}
\end{frame}
